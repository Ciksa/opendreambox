%
% Copyright (c) 2010-2011 Dream Multimedia GmbH, Germany
%                         http://www.dream-multimedia-tv.de/
% Authors:
%   Andreas Oberritter <obi@opendreambox.org>
%
% Permission is hereby granted, free of charge, to any person obtaining a copy
% of this software and associated documentation files (the "Software"), to deal
% in the Software without restriction, including without limitation the rights
% to use, copy, modify, merge, publish, distribute, sublicense, and/or sell
% copies of the Software, and to permit persons to whom the Software is
% furnished to do so, subject to the following conditions:
%
% The above copyright notice and this permission notice shall be included in
% all copies or substantial portions of the Software.
%
% THE SOFTWARE IS PROVIDED "AS IS", WITHOUT WARRANTY OF ANY KIND, EXPRESS OR
% IMPLIED, INCLUDING BUT NOT LIMITED TO THE WARRANTIES OF MERCHANTABILITY,
% FITNESS FOR A PARTICULAR PURPOSE AND NONINFRINGEMENT. IN NO EVENT SHALL THE
% AUTHORS OR COPYRIGHT HOLDERS BE LIABLE FOR ANY CLAIM, DAMAGES OR OTHER
% LIABILITY, WHETHER IN AN ACTION OF CONTRACT, TORT OR OTHERWISE, ARISING FROM,
% OUT OF OR IN CONNECTION WITH THE SOFTWARE OR THE USE OR OTHER DEALINGS IN
% THE SOFTWARE.
%

\documentclass[a4paper]{article}
\usepackage{alltt}
\usepackage{amsmath}
\usepackage{listings}
\usepackage{hyperref}
\usepackage{parskip}
\hypersetup{
    colorlinks,%
    citecolor=black,%
    filecolor=black,%
    linkcolor=black,%
    urlcolor=black
}
%\setlength{\textwidth}{12cm}

\newcommand{\plaintext}[1]{\texttt{\small #1}}
\newcommand{\shell}[1]{\texttt{\small #1}}

\begin{document}
\title{opendreambox}
\author{Andreas Oberritter \shell{<obi@opendreambox.org>}}
\date{March 2012}
\maketitle
%\thispagestyle{empty}
%\pagestyle{empty}
\tableofcontents
\pagebreak

\section{Introduction}
  \begin{flushleft}
    This document describes the OpenDreambox distribution, an embedded Linux
    distribution for Set-Top-Boxes manufactured by \href{http://www.dream-multimedia-tv.de/}{Dream Multimedia GmbH}.

    OpenDreambox is based on \href{http://www.openembedded.org/}{OpenEmbedded} build framework, which
    uses \href{http://bitbake.berlios.de/}{BitBake} to transform build instructions into
    distributable firmare images and software packages.
  \end{flushleft}

  \subsection{Target audience}
    \begin{flushleft}
      Developers familiar with previous versions of OpenDreambox or OpenEmbedded
      in general.
    \end{flushleft}

  \subsection{Supported products}
    \label{products}
    \begin{flushleft}
      The current version includes support for the following machines:

        \begin{tabular}{ | l | l | }
          \hline
          \textbf{Product name} & \textbf{Environment variable} \\ \hline
          & \\
          DM 500 HD & \shell{MACHINE=dm500hd} \\
          DM 800 HD PVR & \shell{MACHINE=dm800} \\
          DM 800 HD SE & \shell{MACHINE=dm800se} \\
          DM 7020 HD & \shell{MACHINE=dm7020hd} \\
          DM 8000 HD PVR & \shell{MACHINE=dm8000} \\
          & \\
          \hline
        \end{tabular}
    \end{flushleft}

  \subsection{License}

    \begin{verbatim}
Copyright (c) 2010-2012 Dream Multimedia GmbH, Germany
                        http://www.dream-multimedia-tv.de/
Authors:
  Andreas Frisch <fraxinas@opendreambox.org>
  Andreas Monzner <ghost@opendreambox.org>
  Andreas Oberritter <obi@opendreambox.org>
  Mladen Horvat <acid-burn@opendreambox.org>
  Stefan Pluecken <thedoc@opendreambox.org>
  Stephan Reichholf <reichi@opendreambox.org>

Permission is hereby granted, free of charge, to any person obtaining a copy
of this software and associated documentation files (the "Software"), to deal
in the Software without restriction, including without limitation the rights
to use, copy, modify, merge, publish, distribute, sublicense, and/or sell
copies of the Software, and to permit persons to whom the Software is
furnished to do so, subject to the following conditions:

The above copyright notice and this permission notice shall be included in
all copies or substantial portions of the Software.

THE SOFTWARE IS PROVIDED "AS IS", WITHOUT WARRANTY OF ANY KIND, EXPRESS OR
IMPLIED, INCLUDING BUT NOT LIMITED TO THE WARRANTIES OF MERCHANTABILITY,
FITNESS FOR A PARTICULAR PURPOSE AND NONINFRINGEMENT. IN NO EVENT SHALL THE
AUTHORS OR COPYRIGHT HOLDERS BE LIABLE FOR ANY CLAIM, DAMAGES OR OTHER
LIABILITY, WHETHER IN AN ACTION OF CONTRACT, TORT OR OTHERWISE, ARISING FROM,
OUT OF OR IN CONNECTION WITH THE SOFTWARE OR THE USE OR OTHER DEALINGS IN
THE SOFTWARE.
    \end{verbatim}

  \subsection{Obtaining the source code}
    OpenDreambox uses the Git version control system. To obtain the source
    code, it is required to install Git. See \url{http://git-scm.com/}.

    To initially download the source into the current directory, issue the
    following command:

    \shell{git clone git://git.opendreambox.org/git/opendreambox.git}

  \subsection{Quick start}
    For the impatient:

    \shell{make -C opendreambox image}

    If this command fails, prerequisites my be missing. See section \ref{prerequisites}.

  \subsection{Directory structure}
    OpenDreambox consists of a set of layers containing build instructions.
    This meta data is used by BitBake to download and compile source code
    and to assemble installable software packages and firmware images.

    Currently, there are five layers, ordered by priority from lowest to
    highest:

    \begin{itemize}
      \item \shell{openembedded-core}
      \item \shell{meta-openembedded}
      \item \shell{meta-opendreambox}
      \item \shell{meta-bsp/common}
      \item \shell{meta-bsp/\$\{MACHINE\}}
    \end{itemize}

    If a recipe for the same package exists in multiple layers,
    then the higher priority layer takes precedence over the lower priority
    layer.

    For example, \shell{libmad\_0.15.1b.bb} exists in both
    \shell{meta-openembedded} and \shell{openembedded-core}. The recipe in
    \shell{meta-openembedded} will be used, because \shell{openembedded-core}
    has lower priority.

    \textbf{NOTE:} This would still be true even if the version of \shell{libmad} in
    \shell{openembeded-core} was higher than the version in \shell{meta-openembedded},
    unless \shell{PREFERRED\_VERSION\_libmad} was set to the version in \shell{openembedded-core}.
    There is currently no way to prefer a version of a lower priority layer, if the
    same version is present in a higher priority layer.

    \subsubsection{openembedded-core and meta-openembedded}
       These directories contain copies of Git repositories from git.openembedded.org, including
       the OpenEmbedded Core layer ("meta") and the OpenEmbedded layer ("meta-oe"). They get created
       automatically when building the distribution for the first time.

       Throughout this document, the combination of these directories will be
       referred to as OpenEmbedded.

    \subsubsection{meta-opendreambox}
      This directory contains:

      \begin{itemize}
        \item Recipes for packages written specifically for the OpenDreambox distribution
        \item Modifications to recipes from OpenEmbedded
        \item Recipes for software versions older than those available from OpenEmbedded
        \item Recipes for software versions newer than those available from OpenEmbedded
      \end{itemize}

    \subsubsection{meta-bsp}
      The directory \shell{meta-bsp} contains Board Support Packages (BSP)
      for the supported Dreambox models. This includes:

      \begin{itemize}
        \item Hardware drivers
        \item Machine specific overrides
        \item The Linux kernel
        \item The boot loader
        \item Splash images
      \end{itemize}

\section{Prerequisites}
  \label{prerequisites}

  \subsection{Required software}

   The OpenEmbedded project provides a general list of prerequisites for
   many Linux distributions and also for some other operating systems.

   \begin{itemize}
     \item \url{http://www.openembedded.org/index.php/OEandYourDistro}
     \item \url{http://www.openembedded.org/index.php/Required\_software}
   \end{itemize}

   It is highly recommended to use Linux to build OpenDreambox. In theory,
   any recent distribution will do, but not many distributions have been
   verified to build OpenDreambox without errors. Tested distributions include:

   \begin{itemize}
     \item Debian "squeeze" 6.0 (i386)
     \item Debian "wheezy" (i386)
     \item Ubuntu "oneiric" 11.10 (amd64)
   \end{itemize}

   Distributions known to cause problems:

   \begin{itemize}
     \item Arch Linux: make 3.82 causes build failures for some packages.
           Try downgrading to make 3.81.
     \item Ubuntu "lucid" 10.04 LTS (i386): The dynamic linker crashes
           frequently, interrupting the build process.
   \end{itemize}

   Distributions known not to work:

   \begin{itemize}
     \item Debian "lenny" 5.0: Comes with Python 2.5, which is too old.
   \end{itemize}

\pagebreak

\section{Major changes since previous releases}

  \subsection{Changes since release 1.6}
    \begin{itemize}
      \item Recipes were split across multiple layers and categorized.
      \item \shell{env.source} has been replaced by two files,
        \shell{bitbake.env} and \shell{cross-compile.env}. The former sets a minimal
        environment that is needed to execute bitbake. The latter creates
        machine-specific command aliases, in order to compile external software.
      \item All machines share a common \shell{tmp} directory.
      \item \shell{\$\{MACHINE\}/build} directories were renamed to \shell{build/\$\{MACHINE\}}.
      \item Kernel packages were renamed from \shell{linux-\$\{MACHINE\}} to \shell{linux-dreambox}.
      \item Support for machines based on ATI Xilleon or IBM STB was dropped.
    \end{itemize}

\section{Known Bugs}

  None.

\pagebreak

\section{Building OpenDreambox}

  In the top level directory, there is a \shell{Makefile}, which is used to
  set up build directories and to fetch or update all used repositories.
  The Makefile can be influenced by environment variables, either
  by specifing them on the command-line or by storing them in a file called
  \shell{conf/make.conf}. It is not recommended to edit the \shell{Makefile} directly
  in order to avoid conflicts with future updates.

  When the \shell{Makefile} is run for the first time, the following steps will
  be executed:

  \begin{itemize}
    \item Creation of configuration files
      \begin{itemize}
        \item \shell{bitbake.env}
        \item \shell{conf/opendreambox.conf}
        \item \shell{build/\$\{MACHINE\}/conf/bblayers.conf}
        \item \shell{build/\$\{MACHINE\}/conf/local.conf}
      \end{itemize}
    \item Update or checkout of Git repositories
      \begin{itemize}
        \item OpenDreambox
        \item BitBake
        \item OpenEmbedded
      \end{itemize}
  \end{itemize}

  \subsection{Makefile targets}
    Run \shell{make help} to get a list of targets of the top level Makefile, together with a brief description.

  \subsection{Configuration variables}
    \subsubsection{BB\_NUMBER\_THREADS}
      Controls how many BitBake tasks may run at a time. Defaults to the
      number of cores available on the build system. The number of cores is
      obtained from \shell{/proc/cpuinfo}. If this file is unavailable, the
      default value is 1.

    \subsubsection{MACHINE}
      Controls the target machine to build packages for. See section \ref{products}
      for a list of supported products.

    \subsubsection{PARALLEL\_MAKE}
      Controls how many processes per recipe \shell{make} may use. Defaults to
      the number of cores available on the build system. The number of cores is
      obtained from \shell{/proc/cpuinfo}. If this file is unavailable, the
      default value is 1.

      Don't increase this option unless you know what you're doing.

    \subsubsection{RM\_WORK}
      Controls whether the temporary work directory of each package may be
      cleaned automatically after the respective package has been built
      successfully. Valid values are \shell{yes} (default) and \shell{no}.

      If set to \shell{yes}, only log files will be preserved. Setting
      \shell{RM\_WORK} to \shell{no} may be useful to debug packaging errors.

  \subsection{Adding custom layers}
  \label{custom_layers}
     It is possible to add custom layers to the configuration. This can be done globally
     and per machine. To add a layer globally, edit \shell{conf/bblayers-ext.conf}. To
     add a machine-specific layer, edit \shell{conf/bblayers-\$\{MACHINE\}-ext.conf}.
     Assuming that an additional layer is available at \shell{/home/username/custom-layer},
     the entry to add to the file will look like this:

     \shell{BBLAYERS =+ "/home/username/custom-layer"}

  \subsection{Adding custom options}
     It is possible to tweak a lot more options than those used by the
     \shell{Makefile} by editing \shell{conf/local-ext.conf} or \shell{conf/local-\$\{MACHINE\}-ext.conf}.

     For example, if the firmware shall use the package feed built on the develoment machine, which
     happens to be 192.168.1.1 and has a webserver configured to point to \shell{tmp/deploy/ipk},
     a line like the following may be added:

     \shell{DISTRO\_FEED\_URI = "http://192.168.1.1/\$\{DISTRO\}/\$\{DISTRO\_VERSION\}"}

     In general, any variable in OpenEmbedded that uses weak assignment (\shell{?=}) may be
     overridden from those files.

  \subsection{Setting up a build directory}
     To set up a build directory for e.g. \textbf{DM 500 HD} run \shell{make MACHINE=dm500hd}. If
     \shell{MACHINE=dm500hd} has been set in \shell{conf/make.conf}, you can simply run \shell{make}
     with no arguments instead. This will create and initialize the directory \shell{build/dm500hd}.

  \subsection{Building a firmware image}
     To build a firmware image for e.g. \textbf{DM 500 HD} run \shell{make MACHINE=dm500hd image}.
     If \shell{MACHINE=dm500hd} has been set in \shell{conf/make.conf}, you can simply run
     \shell{make image} instead.

  \subsection{Building a package}
     To build a single package, BitBake has to be used directly. First, the environment
     has to be set up, in order to make BitBake available to the shell. This can be done
     with the following command:

     \shell{source bitbake.env}

     BitBake must be run from the machine's build directory. For \textbf{DM 500 HD} run:

     \shell{cd build/dm500hd}

     In order to build enigma2, run:

     \shell{bitbake enigma2}.

\section{Development hints}

  \subsection{Cross-compilation of external software}
    OpenDreambox provides a script called \shell{cross-compile.env}. Once run, the script will create
    the following commands (shell aliases), aiming to ease cross-compilation of external source trees:

    \begin{itemize}
      \item \shell{oe\_autoreconf} - Calls OE's version of \shell{autoreconf}. Useful for projects based on GNU autotools.
      \item \shell{oe\_runconf} - Calls \shell{./configure} with parameters suitable for OE. Useful for projects based on GNU autotools.
      \item \shell{oe\_runmake} - Calls \shell{make} with parameters suitable for OE. Useful for projects based on GNU autotools or GNU make.
      \item \shell{oe\_env} - Useful to execute arbitrary commands in the OE environment (e.g. \shell{oe\_env env}).
      \item \shell{oe\_setenv} - Exports the OE environment to the currently running shell session.
    \end{itemize}

    The script needs to know the target machine, which can be one of the supported products (see \ref{products}). Because the script
    adds aliases to your currently running shell session, it has to be invoked using \shell{source} or \shell{.} (the dot operator), e.g.:

    \shell{source cross-compile.env dm7020hd}

    The script may be called from any location, but must reside inside the OpenDreambox Git tree.
    You may want to create a symlink to \shell{cross-compile.env} somewhere in your \shell{PATH}.

  \subsection{Coding style}
    Run \shell{scripts/do\_stylize.sh} on new recipes to ensure proper coding style.

  \subsection {Package architecture}
    Set \shell{PACKAGE\_ARCH = "\$\{MACHINE\_ARCH\}"} if either condition is met:
    \begin{itemize}
      \item The recipe uses \shell{COMBINED\_FEATURES}, \shell{MACHINE\_FEATURES} or \shell{DREAMBOX\_FEATURES}.
      \item The recipe is part of \shell{meta-bsp}.
    \end{itemize}

\section{Bug reports and patches}

  Please send bug reports and patches to the Enigma2 development mailing list \shell{<enigma2-devel@lists.elitedvb.net>}.

  A comfortable way to create patches is to use \shell{git format-patch}, after all changes have been committed to your
  local copy of the repository.

\end{document}
